\documentclass{article}
\begin{document}
\title{Is Hoax a Cyber Security Problem?}
\author{Budi Rahardjo}
\date{\today} 
% \date{\today} date coulde be today 
% \date{25.12.00} or be a certain date
% \date{ } or there is no date 
\maketitle

\begin{abstract}
A while back, I wrote a paper on cyber security and included a section on hoax.
That section was edited out because it was considered out of topic. Not a cyber
security problem. In this paper, I argue that hoax is a cyber security problem.
\end{abstract}

\section{Introduction}
Cyber security is a hot topic these days. There are numerous incidents (events)
related to cyber security, ranging from (cyber) frauds to (cyber) war.
Statistics are available to support this. However, not many mentioned hoax as a
cyber security issue.

Many view information security as technical problems, such as hacking (or
cracking, the proper term) networks and applications. It should be noted that
there are non-technical issues related to information security. For example, in
business, the failure of IT system can disrupt business. Thus, there is a need for
disaster recovery plan (DRP), business continuity planning (BCP), and so on. In
IT security terminology, this is an availability aspect. The point is, IT is
not a business issue.

In information security, we have a technique (methodology?) called social
engineering to gain information via social approach. This is a prove that
information security is not only technical in nature.

\section{Hoax}
What is a hoax? [Definition]

Hoaxes are distributed to many channels, such as fake news sites, social media
accounts, WhatsApp groups, and other internet-based services. (Text messaging
or SMS is also a means to distributed hoaxes, but they are not popular since
there are costs related to distributing these hoaxes via text messaging.)

Some examples of fake news sites are ... They often use names or domain names
that are similar to popular domains. Could this be considered a typo squatting
problem?

Many people consider hoax not a cyber security problem.

Hoax is a news (social?) problem. Hoax uses news sites (albeit fake ones) to
propagate. Thus, hoax is a journalism problem. Therefore, it should be dealt by
``dewan pers'' (in Indonesia).

Hoax is distributed via IT-related products. Thus this is a cyber security
problem. It is similar to other crimes or frauds comitted through IT systems.
For example, there are scams in auction sites, fake (bank) transactions. These
are conventional crimes carried through IT-system. They are considered as cyber
crimes, cyber security problems.

(But, if a person is comitting a crime through telephone, is it considered a
cyber crime? or conventional crime?)

\section{Dealing with Hoaxes}
There are several approaches to deal with hoaxes. The first aspect is to detect
hoaxes. The next step is what to do after we know that the message is hoax. We
have a choice of removing it, doing nothing, or contributing to the
dissemination of that hoax.

[Detecting hoax.]

Removing hoax. One technical approach to deal with hoaxes is to borrow methods
in dealing with spam, malware (virus), cyber porn, or cyber gambling. One
method is to filter the domain name (DNS) of the site hosting the hoaxes.
Probems related to this approach are (1) who is responsible of judging whethere
it is hoax or not? (2) who manages the filter list, (3) how do ISP or users use
this filter list?

Using filter list to combat hoax is still problematic, since users can easily
create a new domain. A better approach would be to ``follow the money'', as
being used to fight cyber gambling.

In Indonesia, the authority dealing with this problem is in one roof; Ministry
of Communication and Informatics. In theory, it should be easier to create
regulation related to this.

\section{Concluding Remarks}
Hoax is a cyber security problem.

Dealing with hoax can be done using controls used to deal with other security
problems; such as using filtering, and regulation.

What makes a problem, a cyber security problem?


\end{document}
